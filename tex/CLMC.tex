%   Copyright (c) 2020 I.F.F. dos SANTOS (Ismael Damiao)

%   Permission is hereby granted, free of charge, to any person obtaining a copy 
%   of this software and associated documentation files (the “Software”), to 
%   deal in the Software without restriction, including without limitation the 
%   rights to use, copy, modify, merge, publish, distribute, sublicense, and/or 
%   sell copies of the Software, and to permit persons to whom the Software is 
%   furnished to do so, subject to the following conditions:

%   The above copyright notice and this permission notice shall be included in 
%   all copies or substantial portions of the Software.

%   THE SOFTWARE IS PROVIDED “AS IS”, WITHOUT WARRANTY OF ANY KIND, EXPRESS OR 
%   IMPLIED, INCLUDING BUT NOT LIMITED TO THE WARRANTIES OF MERCHANTABILITY, 
%   FITNESS FOR A PARTICULAR PURPOSE AND NONINFRINGEMENT. IN NO EVENT SHALL THE 
%   AUTHORS OR COPYRIGHT HOLDERS BE LIABLE FOR ANY CLAIM, DAMAGES OR OTHER 
%   LIABILITY, WHETHER IN AN ACTION OF CONTRACT, TORT OR OTHERWISE, ARISING 
%   FROM, OUT OF OR IN CONNECTION WITH THE SOFTWARE OR THE USE OR OTHER DEALINGS 
%   IN THE SOFTWARE.

\documentclass[
   % -- opções da classe memoir --
   article,                      % indica que é um artigo acadêmico
   10pt,                         % tamanho da fonte.
   openright,                    % capítulos começam em pág ímpar (insere página vazia caso preciso).
   oneside,                      % para impressão em recto. Oposto a toweside.
   a4paper,                      % tamanho do papel. 
   sumario = tradicional,        % Para indentar os sumários
	% -- opções da classe abntex2 --
	chapter=TITLE,		% títulos de capítulos convertidos em letras maiúsculas
	section=TITLE,		% títulos de seções convertidos em letras maiúsculas
	subsection=TITLE,	% títulos de subseções convertidos em letras maiúsculas
	subsubsection=TITLE,% títulos de subsubseções convertidos em letras maiúsculas
   % -- opções do pacote babel --
   english,                      % idioma adicional para hifenização.
   french,                       % idioma adicional para hifenização.
   spanish,                      % idioma adicional para hifenização.
   brazil,                       % O último idioma é o principal do documento.
   xcolor=table                  % Tabelas coloridadas.
]{abntex2}
% --------------------------------------
% PACOTES
% --------------------------------------
% --------------------------------------
% ETC
% --------------------------------------

\usepackage[T1]{fontenc}               % Selecao de codigos de fonte.
\usepackage[utf8x]{inputenc}           % Codificacao do documento (conversão automática dos acentos).
\usepackage{ucs}                       % Complemento do anterior.
\usepackage{indentfirst}               % Indenta o primeiro parágrafo

\usepackage[cleanup]{gnuplottex}       % Para uso do GNUPlotTex.

\usepackage{graphicx}                  % Inclusão de figuras.
\usepackage{wrapfig}                   % Figuras junto com o texto
\usepackage{float}                     % Para posicionar figuras corretamente.
\usepackage{subcaption}                % Para legendas nas subfiguras.

\usepackage{amssymb}                   % Para exibir símbolos de conjuntos de números (reais, etc...).
\usepackage{amsmath}                   % Para adcionar equações
\usepackage{amsfonts}                  % Fontes para notação matemática. de cada seção.


\usepackage{multirow}                  % Para mesclar  linhas nas tabelas.
\usepackage{microtype}                 % para melhorias de justificação.
\usepackage{lipsum}                    % para geração de dummy text.
\usepackage{fancyhdr}                  % Pemite alterações no cabeçalho e rodapé.
\usepackage{hyperref}                  % Cria formatação automática de PDF.
\usepackage[hyphenbreaks]{breakurl}    % Quebra de linha em url.
\usepackage{setspace}                  % Espaço duplo entre parágrafos.
\usepackage{mdframed}                  % para caixas de texto como na CIP do verso do título
\usepackage{lettrine}				      % letras capitulares

\usepackage{enumitem}


\usepackage[num,overcite]{abntex2cite} % Citações padrão ABNT, em ordem alfabética.
\citebrackets[]                        % Citações com coxetes.

\usepackage{geometry}                          % Permite configurar as margens da página.
\geometry{hmargin={3cm,2cm},vmargin={3cm,2cm}} % Margens conforme padrão ABNT
% ---
% Pacotes de citações
% ---

% --------------------------------------
% Para fluxogramas
% --------------------------------------

\usepackage{tikz}

\usetikzlibrary{shapes.geometric, arrows}
\tikzstyle{startstop} = [rectangle, rounded corners, minimum width=3cm, minimum height=1cm,text centered, draw=black]
\tikzstyle{io} = [trapezium, trapezium left angle=70, trapezium right angle=110, minimum width=3cm, minimum height=1cm, text centered, draw=black]
\tikzstyle{process} = [rectangle, minimum width=3cm, minimum height=1cm, text centered, draw=black]
\tikzstyle{decision} = [diamond, minimum width=3cm, minimum height=1cm, text centered, draw=black]
\tikzstyle{conector} = [circle, minimum width=0.5cm, minimum height=0.5cm, text centered, draw=black]
\tikzstyle{arrow} = [thick,->,>=stealth]
\tikzstyle{line} = [draw, -latex']

% --------------------------------------
% Para notação formal de proposições
% --------------------------------------
\usepackage{amsthm}
\usepackage{thmtools}
\usepackage{tcolorbox}

% Teorema
\declaretheoremstyle[
   headfont=\normalfont\bfseries,
   bodyfont=\normalfont\mdseries,
   notefont=\normalfont\bfseries,
   notebraces={. (\slshape}{\upshape)},
   headpunct={ },
   qed={ }
]{EstiloTeorema}
\declaretheorem[
   name=Teorema,
   style=EstiloTeorema
]{teorema}

% Definicao
\declaretheorem[
   name=Defini\c{c}\~{a}o,
   style=EstiloTeorema
]{definicao}

% Exemplo
\declaretheoremstyle[
   headfont=\normalfont\bfseries,
   bodyfont=\normalfont\mdseries,
   notefont=\normalfont\bfseries,
   notebraces={: }{},
   headpunct={. },
   spaceabove=0em,
   postheadspace=0em,
   spacebelow=0em,
   qed=\textnormal{\textemdash}
]{EstiloExemplo}
\declaretheorem[
   name=Exemplo,
   parent=chapter,
   style=EstiloExemplo
]{exemplo}

\declaretheorem[style=plain,name=Axioma,qed=\textnormal{\textemdash}]{axioma}

% --------------------------------------
% Para notação química
% --------------------------------------

\usepackage{chemfig}
\usepackage[version=3]{mhchem}

\newcommand\setpolymerdelim[2]{\def\delimleft{#1}\def\delimright{#2}}
\def\makebraces[#1,#2]#3#4#5{%
\edef\delimhalfdim{\the\dimexpr(#1+#2)/2}%
\edef\delimvshift{\the\dimexpr(#1-#2)/2}%
\chemmove{%
\node[at=(#4),yshift=(\delimvshift)]
{$\left\delimleft\vrule height\delimhalfdim depth\delimhalfdim
width0pt\right.$};%
\node[at=(#5),yshift=(\delimvshift)]
{$\left.\vrule height\delimhalfdim depth\delimhalfdim
width0pt\right\delimright_{\rlap{$\scriptstyle#3$}}$};}}
\setpolymerdelim()

% --------------------------------------
% Para exibição de código fonte
% --------------------------------------

% see https://en.m.wikibooks.org/wiki/LaTeX/Source_Code_Listings

\usepackage{listings}
\usepackage{xcolor}

% see https://latexcolor.com/
\definecolor{codegreen}{rgb}{0,0.6,0}
\definecolor{codegray}{rgb}{0.5,0.5,0.5}
\definecolor{amaranth}{rgb}{0.9, 0.17, 0.31}

\usepackage{caption}
\DeclareCaptionFont{white}{\color{white}}
\DeclareCaptionFormat{listing}{\hspace*{-0.4pt}\colorbox{gray}{\parbox{\textwidth}{#1#2#3}}}
\captionsetup[lstlisting]{format=listing,labelfont=white,textfont=white}

\renewcommand{\lstlistingname}{Código}

\lstdefinestyle{c}{
    language=C, % the language of the code (can be overrided per snippet)
    backgroundcolor=\color{white}, % choose the background color; you must add \usepackage{color} or \usepackage{xcolor}
    commentstyle=\color{codegray}, % comment style
    keywordstyle=\color{blue}\bfseries,
    stringstyle=\color{amaranth}, % string literal style
    basicstyle=\ttfamily\scriptsize, % the size of the fonts that are used for the code 
    numbers=none, % Not use numbers
    columns=fixed, % Using fixed column width (for e.g. nice alignment)
    breakatwhitespace=false,
    breaklines=true,
    captionpos=t, % Set caption to top
    keepspaces=true,
    showspaces=false,
    showstringspaces=false,
    showtabs=false,
    tabsize=3, % sets default tabsize to 3 spaces
    frame=lrb, % adds a frame around the code
     belowcaptionskip=-1pt,
     xleftmargin=8pt,
     framexleftmargin=8pt,
     framexrightmargin=5pt,
     framextopmargin=0pt,
     framexbottommargin=0pt,
     framesep=0pt,
     rulesep=0pt
}

\lstdefinestyle{f90}{
    language=Fortran, % the language of the code (can be overrided per snippet)
    backgroundcolor=\color{white}, % choose the background color; you must add \usepackage{color} or \usepackage{xcolor}
    commentstyle=\color{codegray}, % comment style
    keywordstyle=\color{codegreen}\bfseries,
    stringstyle=\color{amaranth}, % string literal style
    basicstyle=\ttfamily\scriptsize, % the size of the fonts that are used for the code 
    numbers=none, % Not use numbers
    columns=fixed, % Using fixed column width (for e.g. nice alignment)
    breakatwhitespace=false,
    breaklines=true,
    captionpos=t, % Set caption to top
    keepspaces=true,
    showspaces=false,
    showstringspaces=false,
    showtabs=false,
    tabsize=3, % sets default tabsize to 3 spaces
    frame=lrb, % adds a frame around the code
     belowcaptionskip=-1pt,
     xleftmargin=8pt,
     framexleftmargin=8pt,
     framexrightmargin=5pt,
     framextopmargin=0pt,
     framexbottommargin=0pt,
     framesep=0pt,
     rulesep=0pt
}
% --------------------------------------
% Texto de caixinha cinza
% --------------------------------------

%\usepackage[most]{tcolorbox}

\definecolor{textgray}{rgb}{0.7,0.7,0.7}
\tcbset{
   on line,
   boxsep=4pt,
   left=0pt,
   right=0pt,
   top=0pt,
   bottom=0pt,
   colframe=white,
   colback=textgray,
   fontupper=\normalsize\ttfamily
}

% --------------------------------------
% CONFIGURAÇÕES DE PACOTES
% --------------------------------------
\newcommand{\cqd}{\hfill {\itshape\bfseries Quod erat demonstrandum.}}
\newcommand{\asen}{\operatorname{arcsen}\,}
\newcommand{\sen}{\operatorname{sen}\,}
\newcommand{\deu}{\partial}
\newcommand{\adj}{\operatorname{\,adj}\,}
\renewcommand{\dim}{\operatorname{\,dim}\,}

\newcommand{\R}{\mathbb{R}}
\newcommand{\C}{\mathbb{C}}
\newcommand{\Z}{\mathbb{Z}}
\newcommand{\N}{\mathbb{N}}

\newcommand{\M}{\mathcal{M}}
% Não se importe com essas coisas, no geral, não será preciso mexer nelas
\renewcommand{\ABNTEXchapterfontsize}{\Large}       %Muda o tamanho da fonte de título de capítulo
\renewcommand{\ABNTEXchapterfont}{\rmfamily}        %Muda a fonte do título de capítulo
\renewcommand{\ABNTEXsectionfontsize}{\large}       %Muda o tamanho da fonte de título de seção
\renewcommand{\cftchapterfont}{\rmfamily}
\renewcommand{\cftchapterpagefont}{\normalsize\cftchapterfont}
\renewcommand{\cftsectionfont}{\rmfamily}
\renewcommand{\cftsectionpagefont}{\cftsectionfont}
\setlength\afterchapskip{\lineskip}

\newcommand{\bra}[1]{\langle#1|}
\newcommand{\ket}[1]{|#1\rangle}
\newcommand{\braket}[2]{\langle#1|#2\rangle}
\newcommand{\ketbra}[2]{|#1\rangle\langle#2|}

\newcommand{\eq}[1]{\hyperref[eq:#1]{equa\c{c}\~ao} \ref{eq:#1}}


% --------------------------------------
% Informações de dados para CAPA e FOLHA DE ROSTO
% --------------------------------------
% Nesta parte ainda não são permitidos acentos, segue um dicionário
% cedilha ç = \c{c}, Ç = \c{C}
% agudo á = \'{a}, Á = \'{A}
% agudo é = \'{e}, É = \'{E}
% agudo í = \'{i}, Í = \'{I}
% agudo ó = \'{o}, Ó = \'{O}
% agudo ú = \'{u}, Ú = \'{U}
\titulo{CLMC - Cadeia Linear de Massas Correlacionadas}
\autor{
   I.F.F. dos SANTOS\thanks{\href{mailto:ismael.santos@fis.ufal.br}{ismael.santos@fis.ufal.br}, \imprimirinstituicao.}
}
\local{Maceió}
\instituicao{Grupo de Propriedades de Transporte em Sistemas de Baixa Dimensionalidade}

% informações do PDF
\hypersetup{
   portuguese,
   pdftitle={\imprimirtitulo}, 
   pdfauthor={\imprimirautor},
   pdfcreator={LaTeX with abnTeX2},
   pdfkeywords={PVI, EDO, C99},
   %pagebackref=true,
   bookmarksdepth=4,
   colorlinks=true,            % false: boxed links; true: colored links
   linkcolor=blue,             % color of internal links
   citecolor=blue,             % color of links to bibliography
   filecolor=mageta,          % color of file links
   urlcolor=blue
}
\makeindex

\begin{document}
% ------------------------------------------------------------------------------
% ELEMENTOS PRÉ-TEXTUAIS
% ------------------------------------------------------------------------------
\maketitle

\begin{resumoumacoluna}
   Este texto explica o modelo matemático dos sistemas que o programa
   \texttt{CLMC} é capaz de simular, além de fornecer instruções sobre como
   usar o programa. Devo explicar que este é um programa didático e espero
   que sirva de consulta para aqueles que pretendem escrever um programa
   para simular um sistema físico onde o principal desafio é
   resolver equações diferenciais. Me esforcei para que o programa seja
   portátil de modo que para o usar basta possiur algum compilador
   compatível com a versão publicada em 1999 da linguagem de programação
   \texttt{C}, carinhosamente chamada de \texttt{C99}. Para além das bibliotecas
   padão do \texttt{C99} o programa conta com uma cópia da biblioteca
   \texttt{libdamiao},\cite{libdamiao} que é compilada junto com o programa.
\end{resumoumacoluna}

% ------------------------------------------------------------------------------
% ELEMENTOS TEXTUAIS
% ------------------------------------------------------------------------------
\textual

% -------------------------------------
% Introdução
% -------------------------------------
\section{Introdução}

Sejam uma cadeia linear (clássica) com $N$ partículas, $q_n$ o quanto
a n-ésima partícula se afastou de sua posição de equilíbrio e $p_n$
seu respectivo momento linear, sua energia cinética é
$\displaystyle\frac{p^2_n}{2 m_n}$, onde $m_n$ é a sua massa.
Agora suponha que a partícula está sujeita a um potencial
proporcional ao quadrado da distância entre ela e suas visinhas
da forma
$\frac{1}{4}[\eta_{(2, n-1)}(q_n - q_{n-1})^2 + \eta_{(2, n)}(q_{n+1} - q_n)^2]$,
onde $\eta_{(2, n)}>0$ é um termo de acoplamento, este potencial pode
ser identificado com a \textit{lei de Hooke}, isto é, a força que
as partículas exercem umas sobre as outras é similar à força que uma mola
exerce (como se estivessem presas umas às outras por molas).
Generalizando este conceito para potenciais com o cubo e a quarta potencia temos
finalmente a energia de uma partícula dada por

\begin{equation}
E_n = \frac{p_n^2}{2 m_n}
   + \sum_{i = 2}^{4}
   \frac{1}{2i}\left[\eta_{(i, n-1)}(q_n - q_{n-1})^i + \eta_{(i, n)}(q_{n+1} - q_n)^i\right]
\end{equation}

Dessa forma o sistema é conservativo e sua energia total é dada pelo
hamiltoniano $H = \sum_{n=1}^N E_n$. A partir desse hamiltoniano é
possível escrever as equações de movimento para o sistema, dadas pelas
equações de Hamilton

\begin{align}
\frac{d}{dt} q_n &= \frac{p_n}{m_n}\\
\frac{d}{dt} p_n &=
   \sum_{i = 2}^{4}
   \eta_{(i, n)}(q_{n+1} - q_n)^{i-1} - \eta_{(i, n-1)}(q_n - q_{n-1})^{i-1}
\end{align}

Dessa forma o objetivo principal do programa \texttt{CLMC} é encontrar uma
solução numérica para as equações de Hamilton, isto é, encontrar
os valores das funções $q_n(t)$ e $p_n(t)$ do tempo $t$.
Por padrão o programa
\texttt{CLMC}
considera como condição inicial (no tempo $t=0$) que $q_n(0) = 0$
e $p_n(0) = m_n v_0 \delta_{(n, N/2)}$
onde $\delta_{(i, j)}$ é o delta de Kronecker
e $v_0$ é a velocidade inicial da partícula no meio da cadeia,
isso equivale a injetar energia cinética no sistema, inicialmente em equilíbrio
dinâmico, e resolver as equações nos permite estudar como a anergia se espalha
e como viaja pela cadeia.

Também é importante saber descrever os extremos da cadeia, o programa
\texttt{CLMC} considera que não há nenhuma interação da primeira e da
última partícula com o exterior, isso é modelado fazendo
$q_0 = q_{N+1} = \eta_{(i, 0)} = \eta_{(i, N)} = 0$.

% -------------------------------------
% Parâmetros
% -------------------------------------
\section{Parâmetros}

O sistema físico, tal como foi apresentado na introdução, possui vários
parâmetros dos quais depende a interação e a propagação de energia injetada na
cadeia, mais específicamente, os parâmetros são as massas das partículas
e os termos de acoplamento.

Por padrão o programa
\texttt{CLMC}
considera que os termos de acoplamento são os mesmo para todas as partículas,
dependendo somente do tipo de potencial, isto é,
$\eta_{(i, n)} = \eta_i$.

Quanto às massas, seus valores são definidos com base em um dos sequintes
métodos:

\begin{itemize}[nosep]
   \item Correlações construídas com uma série de Fourier:
   A série é
   \begin{equation}
   V_n = \sum_{i=1}^{N/2} i^{-0.5\alpha}
   \cos\left (\frac{2\pi n i}{N} + \Phi_i\right )
   \end{equation}
   Onde $\Phi_i\in[0,2\pi)$ são números pseudo-aleatórios e $\alpha \geq 0$
   é um parâmetro que controla o quão correlacionada é a série.
   A partir dessa serie definimos as massas como
   \begin{equation}
   m_n = \left\{\begin{aligned}
   M_1&\;\;\; \text{se}&        V_n& < -r_1 \\
   M_2&\;\;\; \text{se}& -r_1 < V_n& < r_2 \\
   M_3&\;\;\; \text{se}&  r_2 < V_n& < r_3 \\
   M_4&\;\;\; \text{se}&        V_n& > r_3
   \end{aligned}\right.
   \end{equation}
   Para algum conjunto de valores $0 < r_2 < r_1 < r_3$.

   \item Correlações construídas com uma outra série: A série é
   \begin{equation}
   V_n = \sum_{i=1}^{N} \frac{\Phi_i}{
   \left(\frac{|i-n|}{\alpha}+1\right)^2}
   \end{equation}
   Onde $\Phi_i\in[-1,1)$ são números pseudo-aleatórios e $\alpha > 0$
   é um parâmetro que controla o quão correlacionada é a série.
   A partir dessa série definimos as massas como
   \begin{equation}
   m_n = \left \{\begin{aligned}
   M_1&\;\;\; \text{se}&        V_n& < -r \\
   M_2&\;\;\; \text{se}&    -r < V_n& < r \\
   M_3&\;\;\; \text{se}&      V_n& > r
   \end{aligned}\right .
   \end{equation}
   Para algum valor $r > 0$.

   \item Correlações construídas com um mapa de Bernoulli:
   Essa é uma sequência iterativa de números iniciada com $V_0 = \Phi$
   onde $\Phi\in[0, 1)$ é um número pseudo-aleatório, a sequência é
   \begin{equation}
   V_n = \left \{\begin{aligned}
   V_{n-1} + 2^{\alpha-1}\,(1-2b) \,\, V_{n-1}^\alpha + b \;\;\;\;\;\;\;\;&
   \;\;\;\text{se} \;\; 0 \leq V_{n-1} < 0.5 \\
   V_{n-1} - 2^{\alpha-1}\,(1-2b) \,\, (1-V_{n-1})^\alpha + b &
   \;\;\;\text{se} \;\; V_{n-1} \geq 0.5
   \end{aligned}\right .
   \end{equation}
   Onde, por padrão, $b=10^{-12}$ e $\alpha \geq 0$
   é um parâmetro que controla o quão correlacionada é a sequência.
   A partir dessa sequência definimos as massas como $m_n = M_0 + V_n$ onde
   $M_0 > 0$ é uma constante pequena introduzida com o objetivo de evitar massas nulas.
\end{itemize}

Note que todas as maneiras de gerar correlações utilizam um parâmetro $\alpha$,
o fator de correlação, de forma que o primeiro estudo que vem à mente é
verificar como a energia inserida se propaga em diferentes sistemas com
diferentes fatores de correlação. Note também que não há grandes dificuldades
em alterar o programa original para que ele também trabalhe com termos
de acoplamento correlacionados.

% -------------------------------------
% Medidas de localização
% -------------------------------------
\section{Medidas de localização}

Tenho dito que a velocidade inicial $v_0$ que colocamos no centro da cadeia
é uma energia cinética que injetamos e o programa resolve as equações
de Hamilton com o objetivo de estudar como a energia se propaga na rede.
Pois bem, como $H(0)$ é a energia total disponível na cadeia então
a fração $f_n$ dessa energia que está na n-ésima partícula no tempo t é
dada por

\begin{equation}
f_n(t) = \frac{E_n(t)}{H(0)}
\end{equation}

\noindent
a partir daí podemos definir $\sigma$ como a quantidade que mede o quanto
a energia se espalhou na cadeia, essa quantidade é o deslocamento médio
quadrático definido por

\begin{equation}
\sigma(t) = \left(\sum_{n=1}^N \left(n - \frac{N}{2}\right)^2 f_n(t) \right)^{1/2}
\end{equation}

\noindent
finalmente, irei definir $Z$ como a quantidade cujo o inverso
estima quantas partículas
contribuem para o transporte de energia e é dada por

\begin{equation}
Z(t) = \sum_{n=1}^N f_n^2(t)
\end{equation}

% -------------------------------------
% As funções do CLMC
% -------------------------------------
\section{As funções do \texttt{CLMC}}

O programa \texttt{CLMC} sorteia os números pseudo-aleatórios usando a função
\texttt{random}
da biblioteca \texttt{\citetitle{libdamiao}}. Todas as demais funções utilizadas
ou são padrão do \texttt{C99} ou são funções do próprio programa \texttt{CLMC}.
As funções que preparam o sistema físico são:

\begin{itemize}[nosep]

\item \texttt{void \_\_massas(int semente);} ---
Executa a função que gera um conjunto de valores correlacionados,
conforme o método e fator de correlação configurados no arquivo
\texttt{CLMC\_config}, e depois define todas as massas da cadeia.
\texttt{semente} é um número positivo que servirá de base para inicializar o
gerador de números pseudo-aleatórios.

\item \texttt{void \_\_posicoes(void);} --- Zera a posição inicial de
cada partícula.

\item \texttt{void \_\_momentos(void);} --- Zera o momento inicial de
cada partícula, exceto o da partícula do meio, localizada no sítio $N/2$,
que deve ser igual à sua respectiva massa vezes uma velocidade inicial
cujo valor é configurado no arquivo \texttt{CLMC\_config}.

\item \texttt{void \_\_acoplamentos(void);} --- Para cada tipo de potencial
(quadrático, cúbico ou quártico)
iguala os termos de acoplamentos
de todas as partículas, o valor do termo de acoplamento para cada potencial
é definido no arquivo \texttt{CLMC\_config}.
\end{itemize}

Após executar as funções acima o programa \texttt{CLMC} calcula a energia
total do sistema, antes de executar os métodos de solução das equações de Hamilton.


% ------------------------------------------------------------------------------
% ELEMENTOS PÓS-TEXTUAIS
% ------------------------------------------------------------------------------
   \postextual
   % ---------------------------------------------------------------------------
   % Referências bibliográficas
   % ---------------------------------------------------------------------------
   \bibliography{reference}

\end{document}

