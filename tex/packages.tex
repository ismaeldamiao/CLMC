% --------------------------------------
% ETC
% --------------------------------------

\usepackage[T1]{fontenc}               % Selecao de codigos de fonte.
\usepackage[utf8x]{inputenc}           % Codificacao do documento (conversão automática dos acentos).
\usepackage{ucs}                       % Complemento do anterior.
\usepackage{indentfirst}               % Indenta o primeiro parágrafo

\usepackage[cleanup]{gnuplottex}       % Para uso do GNUPlotTex.

\usepackage{graphicx}                  % Inclusão de figuras.
\usepackage{wrapfig}                   % Figuras junto com o texto
\usepackage{float}                     % Para posicionar figuras corretamente.
\usepackage{subcaption}                % Para legendas nas subfiguras.

\usepackage{amssymb}                   % Para exibir símbolos de conjuntos de números (reais, etc...).
\usepackage{amsmath}                   % Para adcionar equações
\usepackage{amsfonts}                  % Fontes para notação matemática. de cada seção.


\usepackage{multirow}                  % Para mesclar  linhas nas tabelas.
\usepackage{microtype}                 % para melhorias de justificação.
\usepackage{lipsum}                    % para geração de dummy text.
\usepackage{fancyhdr}                  % Pemite alterações no cabeçalho e rodapé.
\usepackage{hyperref}                  % Cria formatação automática de PDF.
\usepackage[hyphenbreaks]{breakurl}    % Quebra de linha em url.
\usepackage{setspace}                  % Espaço duplo entre parágrafos.
\usepackage{mdframed}                  % para caixas de texto como na CIP do verso do título
\usepackage{lettrine}				      % letras capitulares

\usepackage{enumitem}


\usepackage[num,overcite]{abntex2cite} % Citações padrão ABNT, em ordem alfabética.
\citebrackets[]                        % Citações com coxetes.

\usepackage{geometry}                          % Permite configurar as margens da página.
\geometry{hmargin={3cm,2cm},vmargin={3cm,2cm}} % Margens conforme padrão ABNT
% ---
% Pacotes de citações
% ---

% --------------------------------------
% Para fluxogramas
% --------------------------------------

\usepackage{tikz}

\usetikzlibrary{shapes.geometric, arrows}
\tikzstyle{startstop} = [rectangle, rounded corners, minimum width=3cm, minimum height=1cm,text centered, draw=black]
\tikzstyle{io} = [trapezium, trapezium left angle=70, trapezium right angle=110, minimum width=3cm, minimum height=1cm, text centered, draw=black]
\tikzstyle{process} = [rectangle, minimum width=3cm, minimum height=1cm, text centered, draw=black]
\tikzstyle{decision} = [diamond, minimum width=3cm, minimum height=1cm, text centered, draw=black]
\tikzstyle{conector} = [circle, minimum width=0.5cm, minimum height=0.5cm, text centered, draw=black]
\tikzstyle{arrow} = [thick,->,>=stealth]
\tikzstyle{line} = [draw, -latex']

% --------------------------------------
% Para notação formal de proposições
% --------------------------------------
\usepackage{amsthm}
\usepackage{thmtools}
\usepackage{tcolorbox}

% Teorema
\declaretheoremstyle[
   headfont=\normalfont\bfseries,
   bodyfont=\normalfont\mdseries,
   notefont=\normalfont\bfseries,
   notebraces={. (\slshape}{\upshape)},
   headpunct={ },
   qed={ }
]{EstiloTeorema}
\declaretheorem[
   name=Teorema,
   style=EstiloTeorema
]{teorema}

% Definicao
\declaretheorem[
   name=Defini\c{c}\~{a}o,
   style=EstiloTeorema
]{definicao}

% Exemplo
\declaretheoremstyle[
   headfont=\normalfont\bfseries,
   bodyfont=\normalfont\mdseries,
   notefont=\normalfont\bfseries,
   notebraces={: }{},
   headpunct={. },
   spaceabove=0em,
   postheadspace=0em,
   spacebelow=0em,
   qed=\textnormal{\textemdash}
]{EstiloExemplo}
\declaretheorem[
   name=Exemplo,
   parent=chapter,
   style=EstiloExemplo
]{exemplo}

\declaretheorem[style=plain,name=Axioma,qed=\textnormal{\textemdash}]{axioma}

% --------------------------------------
% Para notação química
% --------------------------------------

\usepackage{chemfig}
\usepackage[version=3]{mhchem}

\newcommand\setpolymerdelim[2]{\def\delimleft{#1}\def\delimright{#2}}
\def\makebraces[#1,#2]#3#4#5{%
\edef\delimhalfdim{\the\dimexpr(#1+#2)/2}%
\edef\delimvshift{\the\dimexpr(#1-#2)/2}%
\chemmove{%
\node[at=(#4),yshift=(\delimvshift)]
{$\left\delimleft\vrule height\delimhalfdim depth\delimhalfdim
width0pt\right.$};%
\node[at=(#5),yshift=(\delimvshift)]
{$\left.\vrule height\delimhalfdim depth\delimhalfdim
width0pt\right\delimright_{\rlap{$\scriptstyle#3$}}$};}}
\setpolymerdelim()

% --------------------------------------
% Para exibição de código fonte
% --------------------------------------

% see https://en.m.wikibooks.org/wiki/LaTeX/Source_Code_Listings

\usepackage{listings}
\usepackage{xcolor}

% see https://latexcolor.com/
\definecolor{codegreen}{rgb}{0,0.6,0}
\definecolor{codegray}{rgb}{0.5,0.5,0.5}
\definecolor{amaranth}{rgb}{0.9, 0.17, 0.31}

\usepackage{caption}
\DeclareCaptionFont{white}{\color{white}}
\DeclareCaptionFormat{listing}{\hspace*{-0.4pt}\colorbox{gray}{\parbox{\textwidth}{#1#2#3}}}
\captionsetup[lstlisting]{format=listing,labelfont=white,textfont=white}

\renewcommand{\lstlistingname}{Código}

\lstdefinestyle{c}{
    language=C, % the language of the code (can be overrided per snippet)
    backgroundcolor=\color{white}, % choose the background color; you must add \usepackage{color} or \usepackage{xcolor}
    commentstyle=\color{codegray}, % comment style
    keywordstyle=\color{blue}\bfseries,
    stringstyle=\color{amaranth}, % string literal style
    basicstyle=\ttfamily\scriptsize, % the size of the fonts that are used for the code 
    numbers=none, % Not use numbers
    columns=fixed, % Using fixed column width (for e.g. nice alignment)
    breakatwhitespace=false,
    breaklines=true,
    captionpos=t, % Set caption to top
    keepspaces=true,
    showspaces=false,
    showstringspaces=false,
    showtabs=false,
    tabsize=3, % sets default tabsize to 3 spaces
    frame=lrb, % adds a frame around the code
     belowcaptionskip=-1pt,
     xleftmargin=8pt,
     framexleftmargin=8pt,
     framexrightmargin=5pt,
     framextopmargin=0pt,
     framexbottommargin=0pt,
     framesep=0pt,
     rulesep=0pt
}

\lstdefinestyle{f90}{
    language=Fortran, % the language of the code (can be overrided per snippet)
    backgroundcolor=\color{white}, % choose the background color; you must add \usepackage{color} or \usepackage{xcolor}
    commentstyle=\color{codegray}, % comment style
    keywordstyle=\color{codegreen}\bfseries,
    stringstyle=\color{amaranth}, % string literal style
    basicstyle=\ttfamily\scriptsize, % the size of the fonts that are used for the code 
    numbers=none, % Not use numbers
    columns=fixed, % Using fixed column width (for e.g. nice alignment)
    breakatwhitespace=false,
    breaklines=true,
    captionpos=t, % Set caption to top
    keepspaces=true,
    showspaces=false,
    showstringspaces=false,
    showtabs=false,
    tabsize=3, % sets default tabsize to 3 spaces
    frame=lrb, % adds a frame around the code
     belowcaptionskip=-1pt,
     xleftmargin=8pt,
     framexleftmargin=8pt,
     framexrightmargin=5pt,
     framextopmargin=0pt,
     framexbottommargin=0pt,
     framesep=0pt,
     rulesep=0pt
}
% --------------------------------------
% Texto de caixinha cinza
% --------------------------------------

%\usepackage[most]{tcolorbox}

\definecolor{textgray}{rgb}{0.7,0.7,0.7}
\tcbset{
   on line,
   boxsep=4pt,
   left=0pt,
   right=0pt,
   top=0pt,
   bottom=0pt,
   colframe=white,
   colback=textgray,
   fontupper=\normalsize\ttfamily
}

% --------------------------------------
% CONFIGURAÇÕES DE PACOTES
% --------------------------------------
\newcommand{\cqd}{\hfill {\itshape\bfseries Quod erat demonstrandum.}}
\newcommand{\asen}{\operatorname{arcsen}\,}
\newcommand{\sen}{\operatorname{sen}\,}
\newcommand{\deu}{\partial}
\newcommand{\adj}{\operatorname{\,adj}\,}
\renewcommand{\dim}{\operatorname{\,dim}\,}

\newcommand{\R}{\mathbb{R}}
\newcommand{\C}{\mathbb{C}}
\newcommand{\Z}{\mathbb{Z}}
\newcommand{\N}{\mathbb{N}}

\newcommand{\M}{\mathcal{M}}
% Não se importe com essas coisas, no geral, não será preciso mexer nelas
\renewcommand{\ABNTEXchapterfontsize}{\Large}       %Muda o tamanho da fonte de título de capítulo
\renewcommand{\ABNTEXchapterfont}{\rmfamily}        %Muda a fonte do título de capítulo
\renewcommand{\ABNTEXsectionfontsize}{\large}       %Muda o tamanho da fonte de título de seção
\renewcommand{\cftchapterfont}{\rmfamily}
\renewcommand{\cftchapterpagefont}{\normalsize\cftchapterfont}
\renewcommand{\cftsectionfont}{\rmfamily}
\renewcommand{\cftsectionpagefont}{\cftsectionfont}
\setlength\afterchapskip{\lineskip}

\newcommand{\bra}[1]{\langle#1|}
\newcommand{\ket}[1]{|#1\rangle}
\newcommand{\braket}[2]{\langle#1|#2\rangle}
\newcommand{\ketbra}[2]{|#1\rangle\langle#2|}

\newcommand{\eq}[1]{\hyperref[eq:#1]{equa\c{c}\~ao} \ref{eq:#1}}
